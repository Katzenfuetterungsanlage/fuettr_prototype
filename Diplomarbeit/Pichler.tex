\chapter{Teil 1 Mechanik}
\section{Einleitung}
\section{Aufgabenstellung}
\subsection{Zielsetzung}
\subsection{Problematik}
\newpage
\section{Konzepte} 



\subsection{Variante 1} 
\subsubsection{Übersicht der Prozessschritte}
\begin{itemize}
\item[1] Füllen des Futtermagazins
\item[2] Führen zur Schneidplatte
\item[3] Schnitt
\item[4] Pressen
\item[5] Entsorgen
\item[6] Füttern
\end{itemize}

\subsubsection{Füllen des Futtermagazins}

Im folgenden Bild wird mithilfe einer Lego-Darstellung gezeigt, wie das Magazin aus verschiedenen Blickwinkeln befüllt aussieht. Hier muss man beachten das die vom Hersteller zu öffneten Seite in Richtung des Schneidewerks zeigt (die schmale Seite mit der Einkerbung).

\begin{figure}[H]
\begin{center}
\includegraphics[width=13cm]{Bilder/Ablauf_1_png/Magazin_Vorne.png}
\caption{Magazin Vorne}
\end{center}
\end{figure}

\begin{figure}[H]
\begin{center}
\includegraphics[width=13cm]{Bilder/Ablauf_1_png/Magazin_Seitlich.png}
\caption{Magazin Seitlich}
\end{center}
\end{figure}

\begin{figure}[H]
\begin{center}
\includegraphics[width=13cm]{Bilder/Ablauf_1_png/Magazin_Oben.jpeg}
\caption{Magazin Oben}
\end{center}
\end{figure}

\newpage 

\subsubsection{Führen zur Schneidplatte}

In diesem Schritt wird mithilfe eines Greifers (dargestellt durch eine Hand) die Packung in richtiger Position gebracht.

\begin{figure}[H]
\begin{center}
\includegraphics[width=13cm]{Bilder/Ablauf_1_png/Magazin_Auszug.jpeg}
\caption{Magazin Auszug}
\end{center}
\end{figure}

\begin{figure}[H]
\begin{center}
\includegraphics[width=13cm]{Bilder/Ablauf_1_png/Magazin_Auszug_2.jpeg}
\caption{Magazin Auszug 2}
\end{center}
\end{figure}

Wie im Bild gezeigt liegt das Katzenfutterpackerl in der richtigen Position und wird mit zwei Magnetzylindern an der Schneidefläche festgehalten.

\begin{figure}[H]
\begin{center}
\includegraphics[width=13cm]{Bilder/Ablauf_1_png/Schneidebereit.jpeg}
\caption{Schneidebereit}
\end{center}
\end{figure}

Endposition des Greifers. Kerbe liegt genau an der richtigen Position. 4 Magnetzylinder halten den Futterbeutel and dieser Position, damit der Beutel während des Schneidens nicht verrutscht.

\begin{figure}[H]
\begin{center}
\includegraphics[width=13cm]{Bilder/Ablauf_1_png/Fertig_Geschnitten}
\caption{Fertig Geschnitten}
\end{center}
\end{figure}

\newpage
\subsubsection{Schnitt}

In der richtigen Position muss man mit 2 scharfen Klinge mit viel Druck die Packung aufschneiden. Eine davon wird and der Schnittfläche angebracht und die andere macht die Schneidbewegung, wobei die beiden aneinander reibenden Kanten in einem Schnitt resultieren. Die Packung kann mit einem Schnitt vollständig geöffnet werden.

Anhand dieses Bildes wird gezeigt wie der Schnitt funktionieren kann.

\begin{figure}[H]
\begin{center}
\includegraphics[width=13cm]{Bilder/Ablauf_1_png/Schnitt}
\caption{Schnitt}
\end{center}
\end{figure} 

\newpage
\subsubsection{Pressen}

Nach dem Aufschneiden wird mit einer Rolle das Sackerl ausgepresst. Dazu werden zuerst die ersten 2 Magnetzylinder gelöst bis die Rolle vorbei ist. Danach werden sie wieder in Position gebracht. Daraufhin werden die anderen beiden gelöst und die Rolle fährt ans Ende.

\begin{figure}[H]
\begin{center}
\includegraphics[width=13cm]{Bilder/Ablauf_1_png/Ausquetschen_1}
\caption{Ausquetschen Beginn}
\end{center}
\end{figure}

\begin{figure}[H]
\begin{center}
\includegraphics[width=13cm]{Bilder/Ablauf_1_png/Ausquetschen_2}
\caption{Ausquetschen Mitte}
\end{center}
\end{figure}

\begin{figure}[H]
\begin{center}
\includegraphics[width=13cm]{Bilder/Ablauf_1_png/Ausquetschen_3}
\caption{Ausquetschen Ende}
\end{center}
\end{figure}
\newpage
\subsubsection{Entsorgen}

Nach dem Auspressen wird die leere Packung durch die Rückklappe in einen Luftdichten Container geworfen. Die Klappe wird durch zwei Stifte gehalten und lässt sich durch ein Scharnier nach hinten klappen. 

\begin{figure}[H]
\begin{center}
\includegraphics[width=13cm]{Bilder/Ablauf_1_png/Auswurf_1}
\caption{Auswurf Beginn}
\end{center}
\end{figure}

Hier im Bild sieht man den Stift der ein vorzeitiges nach Hinten klappen verhindert

\begin{figure}[H]
\begin{center}
\includegraphics[width=13cm]{Bilder/Ablauf_1_png/Auswurf_2}
\caption{Bolzen drinnen}
\label{sad}
\end{center}
\end{figure}


Hier im Bild wurde der Stift entfernt 

\begin{figure}[H]
\begin{center}
\includegraphics[width=13cm]{Bilder/Ablauf_1_png/Auswurf_3}
\caption{Bolzen entfernen}
\end{center}
\end{figure}

Hier im Bild wird demonstriert wie die Magnetzylinder die leere Packung gegen die Klappe drücken, wodurch die Klappe sich öffnet.

\begin{figure}[H]
\begin{center}
\includegraphics[width=13cm]{Bilder/Ablauf_1_png/Auswurf_4}
\caption{Klappe öffnen}
\end{center}
\end{figure}

\begin{figure}[H]
\begin{center}
\includegraphics[width=13cm]{Bilder/Ablauf_1_png/Auswurf_5}
\caption{Fertiger Auswurf}
\end{center}
\end{figure}

\subsubsection{Füttern}

Die Maschine besitzt 5 Futterschüsseln die auf einer drehbaren Platte stehen. Vor dem Füttern wird eine Saubere Platte unter der Stelle, wo später die Packung aufgeschnitten wird, positioniert. Während des Auspressens wird fliegt das Futter in die Futterschüssel. Wenn der Auspressvorgang beendet ist, wird die Futterschüssel an eine Position bewegt, wo die Katze Zugang zum fressen hat.
\newpage


\section{Aufbauten und Tests}

In diesem Abschnitt der Diplomarbeit werden verschiedene Tests der obigen Varianten zu sehen sein. \\

\subsection{Fütterungsexperiment} 

In diesem Experiment wurde getestet wie lange es Dauert bis eine Packung nur mit Hilfe der Schwerkraft ausläuft. Der Beutel wurde nicht extra erwärmt und wird nur an den beiden unteren Ecken gehalten.

\begin{figure}[H]
   \begin{minipage}[hbt]{.4\linewidth} % [b] => Ausrichtung an \caption
      \includegraphics[width=\linewidth]{Bilder/Fuetterungsexperiment/Aufhaengung}
      \caption{Halterung}
   \end{minipage}
   \hspace{.2\linewidth}% Abstand zwischen Bilder
   \begin{minipage}[hbt]{.4\linewidth} % [b] => Ausrichtung an \caption
      \includegraphics[width=\linewidth]{Bilder/Fuetterungsexperiment/Fuetterungs_Anfang}
      \caption{Fütterungs Anfang}
   \end{minipage}
\end{figure}


\begin{figure}[H]
   \begin{minipage}[hbt]{.3\linewidth} % [b] => Ausrichtung an \caption
      \includegraphics[width=\linewidth]{Bilder/Fuetterungsexperiment/Fuetterungs_Mitte}
      \caption{Fütterungs Mitte}
   \end{minipage}
   \hspace{.4\linewidth}% Abstand zwischen Bilder
   \begin{minipage}[hbt]{.3\linewidth} % [b] => Ausrichtung an \caption
     \includegraphics[width=\linewidth]{Bilder/Fuetterungsexperiment/Fuetterungs_Ende}  
      \caption{Fütterungs Ende}
     \label{Fütterungs_Ende}
   \end{minipage}
\end{figure}

In der Abbildung: \ref{Fütterungs_Ende} sieht man das nach 10 Minuten der Inhalte ganz in der Futterschüssel ist, dennoch Tropft es nach.
\newpage
\subsection{Schneideversuch 1.Art der 1.Variante}

Schnitt anhand einer praxischen Anwendung dargestellt. Der Beutel wird mithilfe einer Papierschneidemaschine geschnitten.

\begin{figure}[H]
   \begin{minipage}[hbt]{.3\linewidth} % [b] => Ausrichtung an \caption
      \includegraphics[width=\linewidth]{Bilder/Schneideversuch_1.Art/Einlegen}
      \caption{Einlegen}
   \end{minipage}
   \hspace{.2\linewidth}% Abstand zwischen Bilder
   \begin{minipage}[hbt]{.5\linewidth} % [b] => Ausrichtung an \caption
      \includegraphics[width=\linewidth]{Bilder/Schneideversuch_1.Art/Anfangsschnitt}
      \caption{Anfangsschnitt}
   \end{minipage}
\end{figure}

\begin{figure}[H]
\begin{center}
\includegraphics[width=7cm]{Bilder/Schneideversuch_1.Art/Endschnitt}
\caption{Endschnitt}
\end{center}
\end{figure}
\newpage
\subsection{Schneideversuch 2.Art der 1.Variante}

Mit einem Metallwerkzeug mit Wellenschliffartiger Kante wird der Futterbeutel entlang der Oberseite aufgeschnitten. Um die Packung vollständig geöffnet zu haben, mussten mehrere Schnitte verwendet werden.\\

\begin{figure}[H]
   \begin{minipage}[hbt]{.3\linewidth} % [b] => Ausrichtung an \caption
      \includegraphics[width=\linewidth]{Bilder/Schneideversuch_2.Art/Schneidemittel}
      \caption{Schneidemittel}
   \end{minipage}
   \hspace{.4\linewidth}% Abstand zwischen Bilder
   \begin{minipage}[hbt]{.3\linewidth} % [b] => Ausrichtung an \caption
      \includegraphics[width=\linewidth]{Bilder/Schneideversuch_2.Art/Anfangsschnitt}
      \caption{Anfangsschnitt 2.Art}
      \label{Nach 3 Schnitten}
   \end{minipage}
\end{figure}

In der Abbildung: \ref{Nach 3 Schnitten} erkennt man wie offen die Packung nach 3 Schnitten ist.

\begin{figure}[H]
   \begin{minipage}[hbt]{.4\linewidth} % [b] => Ausrichtung an \caption
      \includegraphics[width=\linewidth]{Bilder/Schneideversuch_2.Art/Mittelschnitt}
      \caption{Mittelschnitt 2.Art}
      \label{Nach 6 Schnitten}
   \end{minipage}
   \hspace{.2\linewidth}% Abstand zwischen Bilder
   \begin{minipage}[hbt]{.4\linewidth} % [b] => Ausrichtung an \caption
      \includegraphics[width=\linewidth]{Bilder/Schneideversuch_2.Art/Endschnitt}
      \caption{Endschnitt 2.Art}
      \label{Nach 9 Schnitten}
   \end{minipage}
\end{figure}
In der Abbildung: \ref{Nach 6 Schnitten} erkennt man wie offen die Packung nach 6 Schnitten ist.\\

In der Abbildung: \ref{Nach 9 Schnitten} wurde die Packung nach 9 Schnitten vollständig geöffnet.

\section{Vergleich der Varianten}
\subsection{Klemmen}
\subsubsection{Einfache Klemme}

\begin{wrapfigure}{r}{0.5\textwidth}
\vspace{-20pt}
  \begin{center}
    \includegraphics[width=0.32\textwidth]{Bilder/Powerpoint/Einfach_Klemme}
  \end{center}
  \caption{Einfache Klemme}
  \label{Einfache Klemme}
  \vspace{-30pt}
\end{wrapfigure}

Die einfach Klemme ist für gewöhnliche Verpackungen gut zu nutzen jedoch ist sie für unsere Variante nicht zu gebrauchen durch ihr Plastikmaterial drück sie die Packung an manchen Stellen zu wenig zusammen und an diesen Stellen kann Flüssigkeit austreten. Außerdem hält sie bei Zugbelastung nur wenig stand. Siehe Abbildung: \ref{Einfache Klemme}

\subsubsection{Hebel Klemme} 

\begin{wrapfigure}{r}{0.5\textwidth}
\vspace{-30pt}
  \begin{center}
    \includegraphics[width=0.32\textwidth]{Bilder/Powerpoint/Hebel_Klemme}
  \end{center}
  \caption{Hebel Klemme}
  \label{Hebel Klemme}
  \vspace{-80pt}
\end{wrapfigure}

Die Hebel Klemme ist für diese Diplomarbeit die bevorzugte Methode sie kann viel Druck auf die Packung ausüben sodass keine Flüssigkeit entrinnen kann. Außerdem lässt sich durch den Hebel mit wenig Kraft die Klemme öffnen. Weiters können die Klemmen auf einer Stange aufgesammelt werden und liegen somit nicht irgendwo in der Maschine. Siehe Abbildung:    
 \ref{Hebel Klemme}
\vspace{+40pt}

\subsubsection{Gummiband Klemme}
 
\begin{wrapfigure}{r}{0.5\textwidth}
\vspace{-40pt}
  \begin{center}
    \includegraphics[width=0.30\textwidth]{Bilder/Powerpoint/Gummiband_Klemme}
  \end{center}
  \caption{Gummiband Klemme}
  \label{Gummiband Klemme}
  \vspace{-20pt}
\end{wrapfigure}

Die Gummiband Klemme hat eine starke Klemmkraft, dies Schützt vor dem Aufplatzen der Verpackung. Das Problem dieser Variante ist das das Gummiband spröder werden kann und irgendwann reißen, also ein hoher Verschleiß. Die Klemmen kann man auch nicht kontrolliert sammeln und somit sind sie schwerer zugänglich.

\subsection{Futterschüsseln}

\subsubsection{Drehfutterplatte}

\begin{wrapfigure}{r}{0.5\textwidth}
\vspace{-40pt}
  \begin{center}
    \includegraphics[width=0.25\textwidth]{Bilder/Powerpoint/Drehplatte}
  \end{center}
  \caption{Drehplatte}
  \label{Drehplatte}
  \vspace{-20pt}
\end{wrapfigure}

Die Drehplatte besteht aus fünf Schüsseln man kann pro Schüssel die Katze 2-mal am Tag füttern abends und morgens. Dadurch hat die Katze jeden Tag einen neue Schüssel und falls sie nicht frisst muss sie nicht Hunger leiden. Auf einer Welle wird eine Platte befestigt 
darin werden fünf Löcher geschnitten und die Schüssel hinein gelegt. Die Drehplatte wird mit einen Schneckengewinde in die gewünschten Position gebracht. Siehe Abbildung:  	
 \ref{Drehplatte} \vspace{+80pt}
 

\subsubsection{Futterplatte Zylinder}

\begin{wrapfigure}{}{0.5\textwidth}
\vspace{-50pt}
  \begin{center}
    \includegraphics[width=0.32\textwidth]{Bilder/Powerpoint/Platte_Zylinder}
  \end{center}
  \caption{Platte Zylinder}
  \label{Platte Zylinder}
  \vspace{-20pt}
\end{wrapfigure}

Die Futterplatte mit Zylinder ist die umständlichste Variante. Es ist eine viereckige Platte auf der Schienen für das schieben der Futterschüsseln platziert sind. Diese werden von Magnetzylindern angeschoben. Der Nachteil hierbei ist, man benötigt viele Bauteile und alle Zylinder müssen zugleich arbeiten um die Futterschüssel zur richtigen Position zu führen. Siehe Abbildung: \ref{Platte Zylinder}



\newpage
\subsubsection{Platte mit einer Schüssel}

\begin{wrapfigure}{r}{0.5\textwidth}
\vspace{-40pt}
  \begin{center}
    \includegraphics[width=0.17\textwidth]{Bilder/Powerpoint/Einschuessel_platte}
  \end{center}
  \caption{Drehplatte}
  \label{Schüssel Eins}
  \vspace{-20pt}
\end{wrapfigure}

Die Platte mit nur einer Schüssel ist leicht zu realisieren da sie nur wenige Bauteile benötigt. Das wäre zum Einem die Platte auf der die Futterschüssel mit einer Schiene darauf platziert ist. Sowohl als auch die zwei Magnetzylinder die die Futterschüssel in die Anfangs und Endposition bringt. Jedoch ein großer Nachteil weswegen diese Methode nicht in Frage kommt ist, wenn die Katze nach dem Füttern nicht frisst dann bleibt der Inhalt in der Schale und trocknet ein oder es kommt Ungeziefer hinein. Das hat zu Folge das die Schüssel jeden Tag befüllt wird und übergeht. Siehe Abbildung: \ref{Schüssel Eins}


\subsection{Futtermagazine}

\subsubsection{Futtermagazin Horizontal}

\begin{wrapfigure}{r}{0.5\textwidth}
\vspace{-40pt}
  \begin{center}
    \includegraphics[width=0.17\textwidth]{Bilder/Powerpoint/Futtermagazin_horizontal}
  \end{center}
  \caption{Futtermagazin Horizontal}
  \label{Schüssel Eins}
  \vspace{-20pt}
\end{wrapfigure}



\section{Konstruktion der Wahlvariante und Details}
\section{Berechnung und Dimensionierung}
\section{Simulation}
\section{Bedienung und Wartung}
\section{Selbstkritische Analyse und Ausblick}

