\ihead{Julian Wolf}
\chapter{Elektronik und Mechanik}
\label{sec:elektronik-und-mechanik}

\section{Anforderung Elektronik Variante 1}
\subsection{Aktoren}
Die Aufgabe der Aktoren im elektrischen Teil besteht darin, ein Förderband und eine Drehplatte anzutreiben. Für beide Anwendungen werden Motoren mit einem hohen Drehmoment und einer niedrigen Drehzahl benötigt. Weiteres dürfen sich die Motoren nicht drehen, wenn sie nicht angesteuert werden.
Zur Positionierung, Befestigung und Durchführung anderer linearer Bewegungen werden Hubmagnete in Form von Zylindermagneten benötigt. Diese haben die Aufgabe Futterbeutel zu verschieben, zu positionieren, festzuhalten und einen Scherenhebel zum öffnen der Futterbeutel zu betätigen.  
\subsection{Sensoren}
Die Aufgabe der Sensoren im elektrischen Teil besteht darin, zu erkennen, ob ein Objekt vor ihnen steht. Dadurch kann überprüft werden, ob Futterschüsseln und Futterbeutel an der richtigen Stelle stehen.
\subsection{Ansteuerungen}
Die Aufgabe der Ansteuerung ist es, mithilfe eines Raspberrys und einem Arduino Nano, Motoren anzusteuern so wie Sensoren abzufragen. 
\newpage

\section{Anforderung Elektronik Variante 2}
\subsection{Aktoren}
Die Aufgabe der Aktoren im elektrischen Teil besteht darin, ein Förderband und eine Drehplatte anzutreiben. Für beide Anwendungen werden Motoren mit einem hohen Drehmoment und einer niedrigen Drehzahl benötigt. Zusätzlich dürfen sich die Motoren nicht drehen, wenn sie nicht angesteuert werden.
\subsection{Sensoren}
Die Aufgabe der Sensoren im elektrischen Teil besteht darin, zu erkennen, ob ein Objekt vor ihnen steht. Dadurch kann überprüft werden, ob Futterschüsseln und Futterbeutel an der richtigen Stelle stehen.
\subsection{Ansteuerungen}
Die Aufgabe der Ansteuerung ist es, mithilfe eines Raspberrys und einem Arduino Nano, Motoren anzusteuern so wie Sensoren abzufragen. \\

\section{Variante 1}
\subsection{Aktoren}
\subsubsection{Hubmagnete}
\begin{figure}[H] 
\begin{center}

\includegraphics[width=5cm]{Bilder/Bauteile/Hubmagnet}
\caption{Hubmagnet}
\label{Hubmagnet}

\end{center}
\end{figure}
Zylinderhubmagnete bestehen aus einem zylindrischen Dauermagnetkern. Umgeben ist dieser Magnetkern von einer Spule und einem Gehäuse, welches als Schutz vor Festkörpern dient. Wird an die Spule eine Spannung angelegt, fließt ein Strom durch die Spule und erzeugt ein Magnetfeld welches den Magnetkern anzieht oder abstößt. Durch eine gezielte Übersteuerung des Magnetzylinders kann die Kraft des Magnetfeldes kurzzeitig verstärkt werden, jedoch verhindert die erhöhte Hitzeentwicklung einen dauerhaften Betrieb im übersteuerten Zustand. Der übersteuerte Zustand wird mit einer geringeren relativen Einschaltdauer gegenüber der Zykluszeit bewirkt. \\Der Strom, der durch die Spule fließt, kann sich durch die kurze Einschaltdauer nicht einpendeln was der Spule erlaubt mit einem durchschnittlich höheren Strom betrieben zu werden, das wiederum zu einem stärkeren Magnetfeld und somit zu einer stärkeren Kraft führt. Die genauen Werte der Kraft zur relativen Einschaltdauer lassen sich dem Datenblatt jedes einzelnen Hubmagneten entnehmen. Ein Vorteil von Hubmagneten ist es, dass sie in allen möglichen Formen, Hublängen, Hubkräften und Wechsel- oder Gleichspannungsbereichen verfügbar sind. Hubmagnete können mit drei verschiedenen Hubzuständen erworben werden. Monostabile Hubmagnete halten sich mithilfe einer Feder bei stromlosem Zustand immer an der Anfangs oder Endposition. Bistabile Hubmagnete verweilen im stromlosem Zustand an ihrer aktuellen Position, also entweder im eingefahrenen oder ausgefahrenen Zustand. Tristabile Hubmagnete bleiben, wie bistabile Hubmagnete, stromlos an ihrer aktuellen Position. Jedoch kann diese Hubmagnetart auch an einer vorgegebenen Mittelposition verweilen.

\subsubsection{Motoren}
Nach einigen Problemen und Variantenänderungen in der Mechanik, wurde die Motorenauswahl erst in Variante 2 vorgenommen, da die Auswahl für Variante 1 sinnfrei wäre.

\subsection{Sensoren}
\begin{figure}[H] 
\begin{center}

\includegraphics[width=5cm]{Bilder/Bauteile/Sensor}
\caption{Sensor}
\label{Sensor}

\end{center}
\end{figure}
Die Auswahl des Sensors für unsere Anwendung wurde durch folgende Kriterien beeinflusst: Befestigungsmöglichkeiten, Preis und zusätzlich benötigte Ansteuerungen.
Kapazitive und induktive Näherungsschalter lassen sich einfach befestigen, sind jedoch in der Anschaffung teuer. Die Reichweite der Sensoren ist nicht sehr groß, was aber für unsere Anwendung nicht relevant ist. Für diese Arten von Sensoren werden zusätzliche Ansteuerungen benötigt. Diese Ansteuerungen sind meistens etwas teuerer und benötigen zur Kommunikation mit einem Raspberry eine Schnittstelle.
Aus diesen Gründen haben wir uns in diesem Anwendungsbereich für einen optischen Näherungsschalter entschieden. Dieser beinhaltet ein fertiges Modul bestehend aus einer Infrarot LED und einem Infrarot Fototransistor. Dieses Modul ist preiswert und leicht zu befestigen. \\Dadurch, dass unsere Anwendung keine genaue Abstandsmessung erfordert, wird keine zusätzliche Ansteuerung benötigt. Zur Stromversorgung reichen 5V Gleichspannung mit einem laut Datenblatt berechneten Vorwiderstand. Die Auswertung erfolgt durch einen digitalen Input-Pin des Raspberrys. Sobald sich eine Reflektorfläche vor dem Sensor bewegt, wird der Fototransistor niederohmig und ermöglicht einen Stromfluss. Dadurch liegt eine Spannung von 3,3V am Input-Pin an und wird vom Raspberry als HIGH-Zustand angesehen. Wenn die Reflektorfläche vom Modul nicht mehr erfasst wird, sperrt der Fototransistor, der Stromfluss wird unterbunden und am Input-Pin liegen nur mehr 0V an, was als LOW-Zustand angesehen wird. \\

Kennlinie des Sensors:

\begin{figure}[H]
  \begin{minipage}[hbt]{0.45\textwidth}
    \includegraphics[width=0.9\textwidth]{Bilder/Kennlinien/Sens_Vf_If}
 	\caption{Diodenspannung zu \\Diodenstrom}
  	\label{Sens_vf_if}
  \end{minipage}
\hspace{.03\linewidth}
  \begin{minipage}[hbt]{0.45\textwidth}
    \includegraphics[width=0.9\textwidth]{Bilder/Kennlinien/Sens_op_If}
  	\caption{Optische Übertragungs-\\stärke zu Diodenstrom}
  	\label{Sens_op_if}
  \end{minipage}
\end{figure}

\subsection{Ansteuerungen}
Da Variante 1 nur kurze Zeit verfolgt wurde, wurden genaue Ansteuerungen für Motoren und Hubmagnete nicht herausgesucht. Der Sensor benötigt keine zusätzliche Ansteuerung.
\newpage
\section{Variante 2}
\subsection{Aktoren}
\subsubsection{Hubmagnete}
Für die zweite Variante werden keine Hubmagnete benötigt.

\subsubsection{Motoren}
\begin{figure}[H] 
\begin{center}

\includegraphics[width=8cm]{Bilder/Bauteile/Motor}
\caption{Motor}
\label{Motor}

\end{center}
\end{figure}
Die Motoren wurden mithilfe folgender Kriterien ausgesucht: Preis, Drehzahl, Kraft, stromloses Verhalten und Spannung.
Wir haben uns  darauf festgelegt nur mit sicherer niedriger Gleichspannung zu arbeiten. Daher kommen nur Motoren mit einem Spannungsbereich von 0-24V in Frage. Aus diesem Grund werden die Auswahl eines Asynchronmotors ausgeschlossen. Die Anwendung fordert, dass sich die Motoren im stromlosem Zustand nicht bewegen lassen. Dadurch können folgende Motorarten ausgeschlossen werden: Synchronmotoren, reine Gleichstrommotoren, Schrittmotoren und Servomotoren, falls diese Motoren nicht über eine aktive Bremse verfügen. Ideal für unsere Anwendungen sind Gleichstrom-Schneckengetriebe-Motoren. Diese Motoren lassen sich aufgrund des mechanischen Aufbaus im stromlosem Zustand nicht drehen. Sie verfügen meistens auch über ein Getriebe, welches eine Übersetzung in eine niedrigere Drehzahl zur Folge hat. Gleichzeitig wird mit diesem Getriebe auch das Drehmoment erhöht, wodurch der Motor eine höhere Kraft aufbringen kann. Da Schneckengetriebemotoren in der Automobil Industrie als Scheibenwischermotoren verwendet werden, sind diese sehr preisgünstig zu erwerben.

\subsubsection{Lüfter}
\begin{figure}[H] 
\begin{center}

\includegraphics[width=5cm]{Bilder/Bauteile/Luefter}
\caption{Lüfter}
\label{Luefter}

\end{center}
\end{figure}
Zur Kühlung des Raspberrys, findet ein externer 12V-Lüfter Verwendung, welcher einen Luftstrom um den Raspberry und den \ac{uC}ontroller erzeugt.
\subsection{Sensoren}
Der ausgewählte optische Sensor wurde in Variante 1 bereits beschrieben. Für genauere Informationen lesen Sie bitte in Punkt 4.3.2 nach.
Es werden zwei dieser Sensoren zur Positionierung verwendet. Der Sensor muss nur erkennen, ob sich eine Reflektorfläche vor ihm befindet. Bei der Futterschüsseldrehplatte wird am Rand der Scheibe, an der Position jeder Futterschüssel ein Reflektorstreifen verwendet. Dadurch kann mit der Software gezählt und erkannt werden, welche Futterschüssel sich aktuell vor dem Sensor befindet. Im Futterbeutelmagazin wird der Sensor dazu verwendet, um erkennen zu können, ob ein Futterbeutel für die Fütterung verfügbar ist. Wird vom Sensor, nach Drehen des Motors, kein weiterer Futterbeutel erkannt, wird von der Software, falls nötig, ein Fehler ausgegeben und der Benutzer informiert. Der Reflektorstreifen wird auf der Verschlussklemme jedes Futterbeutels befestigt.
\subsection{Ansteuerung}
Der Gesamtschaltplan wurde in ProfiCad realisiert. Der Schaltplan beinhaltet die Ansteuerung der Motoren und Sensoren, sowie die Visualisierung der Verkabelung der einzelnen Elemente.
Für ein einfacheres Verständnis wird der Gesamtschaltplan in kleinere Elemente aufgeteilt und in Unterpunkten beschrieben.
\newpage
\subsubsection{Motoransteuerung Variante 1}
Der Gleichstrom Schneckengetriebemotor in unserer Anwendung kann in beide Richtungen betrieben werden, je nach Polarität der Anschlüsse. Um eine softwaretechnisch einfache Ansteuerung ermöglichen zu können, wird oft eine H-Brücke zur Ansteuerung der Motoren verwendet. Diese H-Brücke besteht aus vier Transistoren. Für Motoren mit höheren Stromanforderungen werden MOSFET-Transistoren verwendet. Die Schaltung besteht aus zwei P-Kanal MOSFET T1, T3 und aus zwei N-Kanal MOSFET T2, T4. Die Schaltung laut Abbildung \ref{HBridge}, wurde aus dem Web\footfullcite{HBridge} entnommen und abgeändert. Dabei wurden die Bauteilwerte der MOSFETs und Teile der Schaltung übernommen. Der Raspberry steuert über Optokoppler die MOSFETs. Dabei wird über eine entsprechende Vorschaltung entweder das Gate vom MOSFET auf Ground- oder 12V-Potenzial gezogen. Zur Veranschaulichung wurden in Abbildung \ref{HBridge} die Raspberry Pins mit einer Klemme ersetzt, wobei die Klemme 1, T1, Klemme 2, T2, Klemme 3, T3 und Klemme 4, T4 steuert. 

Es gibt bei dieser Ansteuerungsart folgende erwünschte Zustände der Klemmen: \\
\begin{table}[htb]
\centering
\begin{tabular}{|c|c|c|c|c|c|} \hline
Klemme & Uhrzeigersinn & gegen Uhrzeigersinn & Auslaufen & Bremsen & Bremsen \\ \hline
1 & HIGH & LOW & LOW & HIGH & LOW  \\ \hline
2 & LOW & HIGH & LOW & LOW & HIGH \\ \hline
3 & LOW & HIGH & LOW & HIGH & LOW \\ \hline
4 & HIGH & LOW & LOW & LOW & HIGH \\ \hline
\end{tabular}
\caption{H-Brückenzustände}
\label{HBridge states}
\end{table}

Jede unterschiedliche Beschaltungsart als die oben angeführten, würde zu einem Kurzschluss führen und muss softwaretechnisch verhindert werden. \\


Kennlinie der Transistoren:

\begin{figure}[H]
  \begin{minipage}[hbt]{0.45\textwidth}
    \includegraphics[width=0.9\textwidth]{Bilder/Kennlinien/P_Kanal}
 	\caption{P-Kanal}
  	\label{Pchannel}
  \end{minipage}
\hspace{.03\linewidth}
  \begin{minipage}[hbt]{0.45\textwidth}
    \includegraphics[width=0.9\textwidth]{Bilder/Kennlinien/N_Kanal}
  	\caption{N-Kanal}
  	\label{Nchannel}
  \end{minipage}
\end{figure}


\begin{figure}[H] 
\begin{center}

\includegraphics[width=15cm]{Bilder/Schaltplan/Schaltplan_HBridge}
\caption{H-Bridge}
\label{HBridge}

\end{center}
\end{figure}

Dieser Schaltplan dient nur als Beispiel-Schaltung und wurde aufgrund der Motoransteuerung Variante 2, nicht mehr weiter verfolgt. Um die Schaltung wirklich benutzen zu können, müsste eine Sicherung gegen einen Kurzschluss eingebaut werden. Für unsere Anwendung müsste eine Drehzahlregelung mittels PWM-Signal implementiert werden.
\newpage
\subsubsection{Motoransteuerung Variante 2}
\begin{figure}[H] 
\begin{center}

\includegraphics[width=6.5cm]{Bilder/Bauteile/Motorsteuerung}
\caption{Motorsteuerung}
\label{Motoransteuerung}

\end{center}
\end{figure}
Aus sicherheits- und kostentechnischen Gründen, wurde von einer manuell gesteuerten H-Brücke auf ein fertiges Motorsteuerungsmodul umgestiegen. Der Vorteil dieser Ansteuerung besteht darin, dass ein Kurzschluss der H-Brücke hardwaretechnisch ausgeschlossen werden kann und damit nicht von der Software abgesichert werden muss. Das Motormodul verfügt außerdem über eine interne Strommessung für jeden der zwei unterstützten Motoren. Diese Messung kann bei Bedarf abgefragt werden. Für unsere Anwendung ist dies jedoch nicht erforderlich. Der Motortreiber unterstützt zwei Motoren, welche mit bis zu 30 Ampere bei 16V betrieben werden können, bevor der IC überhitzt oder andere Bauteile zerstört werden. Die Ansteuerung jedes Motors erfolgt über ein digitales Signal. Motor Nr.1 kann über den Pin AI0 aktiviert (enabled) werden. Dies dient zur Sicherung, dass der Motor nicht aus Versehen betätigt werden kann, solange der AI0 Pin nicht auf HIGH ist. Der enable Pin für Motor Nr.2 ist AI1. Um Motor Nr.1 im Uhrzeigersinn drehen zu lassen, muss der Pin D7 mit einem HIGH Signal und der Pin D8 mit einem LOW Signal angesteuert werden. Um Motor Nr.2 gegen den Uhrzeigersinn drehen zu lassen, muss der Pin D8 mit einem HIGH Signal und der Pin D7 mit einem LOW Signal angesteuert werden. Um den Motor Nr.1 bremsen zu können, muss Pin D8 und D7 mit entweder einem HIGH oder einem LOW Signal angesteuert werden. Für Motor Nr.2 ist die Funktionsweise gleich, jedoch muss der Pin D7 mit Pin D4 und Pin D8 mit Pin D9 ersetzt werden. Die externe Stromversorgung für den IC von 5V erfolgt über den VCC Pin. Die externe Stromversorgung für die Motoren erfolgt über den PWR Pin. Zusätzlich muss der GND Pin noch mit einem der beiden Ground Anschlüsse der externen Stromversorgungen verbunden werden. Motor Nr.1 wird über die Outputs A1 und B1 versorgt. Motor Nr.2 wird über die Outputs A2 und B2 versorgt. Beim Anschluss der Motoren muss auf die Polung geachtet und gegebenenfalls getestet werden, ob sich die Motoren auch wirklich beim Ansteuern der D7 oder D4 Pins im Uhrzeigersinn drehen und nicht gegen den Uhrzeigersinn. Falls beim Motor gekennzeichnet, sollte der Pluspol des Motors mit dem A1 beziehungsweise dem A2 Pin verbunden werden. Der Motortreiber verfügt auch über eine interne Drehzahlregelung, welche extra für beide Motoren über zwei \ac{PWM} Input Pins gesteuert werden kann. Die Drehzahl kann über den Duty Cycle des \ac{PWM} Signals geregelt werden.
\subsubsection{PWM-Signal-erzeugung für Motoransteuerung}
\begin{figure}[H] 
\begin{center}

\includegraphics[width=12cm]{Bilder/PWM/Duty_Cycle}
\caption{PWM-Signal}
\label{PWM_Signal}

\end{center}
\end{figure}
Ein \ac{PWM} Signal ist ein periodisches Rechteck-Signal. Eine Periode wird als Duty Cycle bezeichnet. Dieser Duty Cycle enthält eine Prozentangabe, welche aussagt, wie viel Prozent des Duty Cycles HIGH und wie viel Prozent LOW sind. Ein Signal mit einem Duty Cycle von 100\% sagt aus, dass eine Periodendauer zu 100\% aus einem HIGH Signal und zu 0\% aus einem LOW Signal besteht. Ein Duty Cycle von 25\% sagt aus, dass eine Periode zu 25\% aus einem HIGH Signal und zu 75\% aus einem LOW Signal besteht. Das \ac{PWM} Signal kann von manchen ICs intern aufgenommen und der Duty Cycle ausgemessen werden. Dabei muss jedoch beachtet werden, dass die Frequenz des \ac{PWM} Signals nicht die maximale Schaltfrequenz des ICs übersteigt. In den meisten Fällen wird diese Frequenz, wie bei unserem Motortreiber, im Datenblatt angegeben. Das \ac{PWM} kann entweder über eigene ICs mit \ac{PWM} Generatoren oder \ac{uC}s mit internen \ac{PWM} Generatoren erzeugt werden. Das System muss nur über eine Echtzeitfähigkeit besitzen, damit der Duty Cycle keine Takte beim Zählen überspringen und nicht die Frequenz fehlerhaft werden kann. 
Unser \ac{PWM} Generator wurde mithilfe eines Arduino Nanos realisiert. Mithilfe des \ac{uC} Atmega328P wird der interne Zähler (Timer 2) dazu verwendet, einen Output Pin in den richtigen Abständen zu toggeln (von HIGH auf LOW oder umgekehrt schalten) um ein \ac{PWM} Signal zu erzeugen.


\begin{itemize}
\item verwendete Register:
\begin{itemize}
\item TCCR2A: Im TCCR2A Register wird festgelegt, in welchem Modus der Timer laufen soll und was mit den Pins OC2A und OC2B passieren soll, wenn der der aktuelle Timer-Wert mit einem der beiden Output Compare Register übereinstimmt. Die Bits COM2A1 und COM2A0 setzen im Fast-\ac{PWM} Modus fest, ob der OC2A Pin nicht beeinflusst wird und nicht verbunden ist, vom WGM22 Bit abhängig ist oder bei einem Compare Match, also wenn das Output Compare Register OCR2A mit dem aktuellen Timer Wert übereinstimmt, der OC2A Pin gesetzt und beim Reset des Timers resetet wird (non-inverting Mode), oder bei einem Compare Match der OC2A Pin resetet und beim Reset des Timers gesetzt wird (inverting Mode). \\ Die selbe Funktion haben die Bits COM2B1 und COM2B0, nur dass der Pin OC2B verändert und mit dem Compare Registe OCR2B verglichen wird. \\
Die Bits WGM21 und WGM20 setzen den Modus des Signalgenerators, die vom Timer unterstützt werden. Es gibt grundsätzlich drei Modi, in denen der Signalgenerator laufen kann: CTC, Fast \ac{PWM}, \ac{PWM} Phase Correct. \\
\begin{itemize}
\item Im CTC Modus kann der TOP Wert, zu dem der Counter zählt, immer bei jedem vollständigen Zählvorgang geändert werden. Der Zähler zählt in Form eines Sägezahn Signals und toggled den Pin immer, wenn der Top Wert erreicht wird.\\
\item Im Fast \ac{PWM} Modus zählt der Zähler in Form eines Sägezahnsignals und es wird bei jedem Zählschritt überprüft, ob der aktuelle Wert des Zählers mit dem Output Compare Register übereinstimmt. Was passiert, wenn die Werte übereinstimmen, wird in den COM2xx Bits festgelegt. Die Frequenz des Ausgangssignals am Pin kann nur über einen Prescaler verändert werden. Das bedeutet, dass die Frequenz nur sieben Frequenzen mit dem internen Timer annehmen kann. Das OCR2x Register kann während des Betriebes geändert werden. Damit kann der Duty Cycle während des Betriebes geändert werden. \\
\item Der \ac{PWM} Phase Correct Modus, lässt den Zähler von Bottom zu Top und wieder zu Bottom zählen. Das bewirkt, dass er Zähler in Form eines Dreiecksignals zählt. Somit ist das Ausgangs \ac{PWM} Signal immer symmetrisch um den BOTTOM und TOP Wert des Zählers. Daher bleibt das Signal immer Phasen-korrekt. Der Nachteil dieses Modus ist, dass er nur die halbe Frequenz des Fast \ac{PWM} Modus erreichen kann. Der \ac{PWM} Phase Correct Modus kann auch nur sieben Frequenzen annehmen.\\
\end{itemize} 
\item TCCR2B: Im TCCR2B Register wird für unsere Anwendung der Prescaler festgelegt, um die Frequenz des \ac{PWM}-Signales einzustellen. Dazu werden die Bits 0 bis 2 verwendet. Das sind die Bits CS20, CS21 und CS22. Die Bits FOC2A, FOC2B und WGM22 werden in unserer Anwendung nicht benötigt und können ignoriert werden. \\
\item OCR2A: Das Register OCR2A ist das Output Compare Match Register für den OC2A Pin. Das Register kann einen Wert zwischen 0 und 255 haben, wobei im Fast \ac{PWM} Modus 0 einem Duty Cycle von 0\% und 255 einem Duty Cycle von 100\% entspricht.\\
\item OCR2B: Das Register OCR2B ist das Output Compare Match Register für den OC2B Pin. Das Register kann einen Wert zwischen 0 und 255 haben, wobei im Fast \ac{PWM} Modus 0 einem Duty Cycle von 0\% und 255 einem Duty Cycle von 100\% entspricht.\\
\item DDRD: Im Register DDRD können I/O Pins als Output Pins gesetzt werden. Es muss nur das Bit des entsprechenden Pins gesetzt werden.\\
\item DDRB: Im Register DDRB können I/O Pins als Output Pins gesetzt werden. Es muss nur das Bit des entsprechenden Pins gesetzt werden.\\
\end{itemize}
\item Frequenz berechnen:\\
Die maximal Frequenz des \ac{PWM}-Signals für die Motoransteuerung beträgt 20kHz. Das bedeutet, dass man einen Frequenzwert so nahe wie möglich, aber unter der maximalen Frequenz wählen soll.\\
Berechnen lässt sich die Frequenz des \ac{PWM}-Signals mit folgender Formel:
\begin{align*}
f_{OC2xPWM}=\frac{f_{clk\_I/O}}{N*256} \\
\end{align*} 
f$_{OC2xPWM}$ ist die Frequenz, die das \ac{PWM}-Signal haben wird.\\
f$_{clk_I/O}$ ist die Taktfrequenz des Atmega328P, also 16MHz. \\
N ist der Wert des Prescalers, welcher im TCCR2B Register festgelegt wird. \\

\textbf{Frequenz Berechnung:}\\
$f_{OC2xPWM}$=$\frac{f_{clk\_I/O}}{N*256}$ = $\frac{16000000}{8 \cdot 256}$ = $\underline{7812,5Hz}$
\item Programm für Atmega328P: \\
\begin{lstlisting}[caption=$\mu$C-Programm,style=C]
void app_main (void)
{
  DDRD = (1<<PD3);
  DDRB = (1<<PB3);
  TCCR2A = (1<<COM2A1) | (1<<COM2B1) | (1<<WGM21) | (1<<WGM20);
  TCCR2B = (1<<CS21);
  OCR2A = 0x5f;
  OCR2B = 0x5f;
  
}
\end{lstlisting}
\newpage
\begin{itemize}
\item DDRD: Im DDRD Register wird der PD3 Pin auf Output geschaltet.
\item DDRB: Im DDRB Register wird der PB3 Pin auf Output geschaltet.
\item TCCR2A: Im TCCR2A Register wird der Non-Inverting Fast-\ac{PWM} Modus eingestellt für OC2A und OC2B. 
\item TCCR2B: Im TCCR2B Register wird der Prescaler festgelegt. 
\item OCR2A: Im OCR2A Register wird der Duty Cycle festgelegt. 0 entspricht 0\% und 255 entspricht 100\% 
\item OCR2B: Im OCR2B Register wird der Duty Cycle festgelegt. 0 entspricht 0\% und 255 entspricht 100\% 
\end{itemize}
\end{itemize}

Ein Signal, dass nach dem oben angeführten Programm generiert wird, sieht wie in Abbildung \ref{PWM} aus.
\begin{figure}[H] 
\begin{center}

\includegraphics[width=13cm]{Bilder/PWM/PWM}
\caption{PWM-Signal}
\label{PWM}

\end{center}
\end{figure}
\newpage
\subsubsection{Optokoppler}

Die Optokoppler in unserer Anwendung werden zur Potenzialerhöhung oder Potenzialverminderung von 3,3V auf 5V oder umgekehrt verwendet. Durch eine LED und einen Phototransistor werden die Schaltkreise getrennt. Wenn an der LED eine Spannung anliegt, beginnt sie zu leuchten und erhöht damit die Leitfähigkeit des Phototransistors. Dadurch ermöglicht ein Stromfluss der LED-Schaltung einen Stromfluss in der Phototransistor-Schaltung.

Kennlinie des Optokopplers:

\begin{figure}[H]
  \begin{minipage}[hbt]{0.45\textwidth}
    \includegraphics[width=0.9\textwidth]{Bilder/Kennlinien/Opto_Vf_If}
 	\caption{Diodenstrom zu \\Diodenspannung}
  	\label{Opto_Vf_If}
  \end{minipage}
\hspace{.03\linewidth}
  \begin{minipage}[hbt]{0.45\textwidth}
    \includegraphics[width=0.9\textwidth]{Bilder/Kennlinien/Opto_Vce_Ic}
  	\caption{Transistorstrom zu \\Transistorspannung}
  	\label{Opto_Vce_Ic}
  \end{minipage}
\end{figure}
\subsubsection{Gesamtschaltplan}
\begin{figure}[H]
\begin{minipage}[t]{6cm}
\vspace{0pt}
\centering
\includegraphics[width=8cm]{Bilder/Schaltplan/Blockschaltbild}
\caption{Blockschaltbild}
\label{Blockschaltbild}
\end{minipage}
\hfill
\begin{minipage}[t]{7cm}
\vspace{0pt}
Das Blockschaltbild vereinfacht die Gesamtschaltung und zeigt an, welche Bauteile miteinander verbunden sind. Die einzelnen Bauteile werden in den weiteren Punkten noch genauer beschrieben.
Verbindungen der Schaltpläne erfolgen mittels der Klemmen X1 bis X6.
\end{minipage}
\end{figure}

 
\subsubsection{Verkabelung der Motoransteuerung mit dazugehörigen Elementen}
\begin{figure}[H] 

\begin{center}

\includegraphics[width=15cm]{Bilder/Schaltplan/Motoransteuerung}
\caption{Motoransteuerung}
\label{Motoransteuerung}
\end{center}
\end{figure}
Die Motoransteuerung besteht aus folgenden Komponenten:\\
\begin{itemize}
\item Dem Motortreiber mit dem IC VNH2SP30.
\item Der Klemme X4, X3, X2, X1.
\item Sechs Optokopplern, welche die Ausgangsspannung von den GPIO-Pins auf 5V anheben.
\item Den beiden Motoren M1 und M2.
\item Den Widerständen R1 bis R6, welche den Strom begrenzen, damit der Optokoppler nicht überlastet wird.
\item Den Pulldown-Widerständen R7 bis R12 welche ein floaten, also eine fehlerhafte Spannung an den Input-Pins der Motorsteuerung verhindern sollen. \\
\end{itemize}
\newpage
\textbf{Genereller Ablauf:}
\begin{enumerate}
\item Der Arduino Nano erzeugt ein dauerhaftes \ac{PWM}-Signal, welches an die \ac{PWM}-Inputs D6 und D5, über die Klemme X4, der Motoransteuerung gelegt wird.
\item Die GPIO-Pins geben ein High-Signal über die Klemme X3, von 3,3V aus, welches die Optokoppler ansteuert. 
\item Der Optokoppler schaltet seinen internen Phototransistor durch und ermöglicht einen Stromfluss.
\item Der Strom fließt nun von der 5V Spannungsversorgung über die Klemme X2, über den Optokoppler in den Input-Pin der Motoransteuerung.
\item Je nach Ansteuerung der Pins, dreht sich einer oder beide Motoren im oder gegen den Uhrzeigersinn, oder werden gebremst. Dabei fließt ein Strom von der 12V Spannungsversorgung über die Klemme X1, über die Motoransteuerung zu den Motoren.
\item Die Motoransteuerung regelt die Spannung an den Motoren mittels des \ac{PWM}-Signals und regelt auch die Drehrichtung der Motoren je nach Ansteuerung.
\end{enumerate}
\textbf{Berechnungen:}
\begin{itemize}
\item R1 bis R6: \\
Die Spannung, die abfällt und der Strom, der fließt, kann aus dem Datenblatt des Optokopplers entnommen werden.\\
$U_{Optokoppler 5mA}=$1,15V \\
$U_{Optokoppler 10mA}=$1,2V \\
$I_{Optokoppler}$=5/10mA \\

\begin{center}
$R_{1,2,3,4,5,6}=\frac{U}{I}=\frac{U_{GPIO}-U_{Optokoppler 5mA}}{I_{Optokoppler}}=\frac{3,3V-1,15V}{0,005A}=430\Omega$
\end{center}

\begin{center}
$R_{1,2,3,4,5,6}=\frac{U}{I}=\frac{U_{GPIO}-U_{Optokoppler 10mA}}{I_{Optokoppler}}=\frac{3,3V-1,2V}{0,010A}=210\Omega$
\end{center}

Da der Ausgangsstrom durch den Eingangsstrom bestimmt wird, kann zur vereinfachten Berechnung durch die Kennlinie von Abbildung \ref{Opto_Vce_Ic} ein Eingangsstrombereich angenommen werden. Dadurch kann mithilfe der Kennlinie von Abbildung \ref{Opto_Vf_If} die abfallende Spannung am Optokoppler abgelesen werden. Mithilfe dieser Werte kann ein Widerstandsbereich berechnet werden, in dem der Widerstandswert liegen muss, damit der Ausgangsstrom hoch genug ist. Der berechnete Widerstandsbereich von $R_{1,2,3,4,5,6}$ beträgt 210$\Omega$-430$\Omega$. Es wird für diese Schaltung ein 390$\Omega$ Widerstand ausgewählt.
\end{itemize}

\subsubsection{Verkabelung des Sensors mit dazugehörigen Elementen} 
\begin{figure}[H] 
\begin{center}

\includegraphics[width=15cm]{Bilder/Schaltplan/Sensoransteuerung}
\caption{Sensoransteuerung}
\label{Sensoransteuerung}

\end{center}
\end{figure}

Die Sensoransteuerung besteht aus folgenden Komponenten:
\begin{itemize}
\item Den optischen Sensoren OPB732WZ.
\item Der Klemme X2, X5, X6. 
\item Zwei Optokopplern, welche die Spannung der Sensoren von 5V auf ein 3,3V Potenzial bringen.
\item Den Widerständen R13 und R14, welche einen Spannungsabfall erzeugen, damit der Sensor nicht überlastet wird.
\item Den Widerständen R17 und R18, welche einen Spannungsabfall erzeugen, damit der Optokoppler nicht überlastet wird.
\item Den Pulldown-Widerständen R15 und R16 welche ein floaten, also eine fehlerhafte Spannung an den Input-Pins des Raspberrys verhindern sollen.\\
\end{itemize}
\newpage
\textbf{Genereller Ablauf:}
\begin{enumerate}
\item Wenn der Sensor eine Reflektor-Fläche vor sich erkennt, schaltet der Phototransistor im Sensor durch.
\item Der durchgeschaltete Phototransistor ermöglicht einen Stromfluss von der Spannungsquelle über die Klemme X2, in den Optokoppler.
\item Dadurch schaltet der interne Phototransistor des Optokopplers durch.
\item Der durchgeschaltete Phototransistor ermöglicht einen Stromfluss von einem der 3,3V-Pins über die Klemme X5, in einen der GPIO-Pins, über die Klemme X6 des Raspberrys.
\item Somit kann erkannt werden, welcher der Sensoren eine Reflektor-Fläche erkennt und weitere softwaretechnische Maßnahmen können erfolgen.
\end{enumerate}
\textbf{Berechnungen:}
\begin{itemize}
\item R17 und R18: \\
Die Spannung, die abfallen muss und der Strom, der fließt, kann aus den Datenblättern des Sensors und des Optokopplers entnommen werden.\\
$U_{Optokoppler 5mA}=$1,15V \\
$U_{Optokoppler 10mA}=$1,2V \\
$U_{CE(SAT)}=$0,4V \\
$I_{Optokoppler}$=5/10mA \\

\begin{center}
$R_{17,18}=\frac{U}{I}=\frac{U_{5V}-U_{Optokoppler 5mA}-U_{CE(SAT)}}{I_{Optokoppler}}=\frac{5V-1,15V-0,4V}{0,005A}=690\Omega$
\end{center}

\begin{center}
$R_{17,18}=\frac{U}{I}=\frac{U_{5V}-U_{Optokoppler 10mA}-U_{CE(SAT)}}{I_{Optokoppler}}=\frac{5V-1,2V-0,4V}{0,01A}=340\Omega$
\end{center}

Da der Ausgangsstrom durch den Eingangsstrom bestimmt wird, kann zur vereinfachten Berechnung durch die Kennlinie von Abbildung \ref{Opto_Vce_Ic} ein Eingangsstrombereich angenommen werden. Dadurch kann mithilfe der Kennlinie von Abbildung \ref{Opto_Vf_If} die abfallende Spannung am Optokoppler abgelesen werden. Mithilfe dieser Werte kann ein Widerstandsbereich berechnet werden, in dem der Widerstandswert liegen muss, damit der Ausgangsstrom hoch genug ist. Der berechnete Widerstandsbereich von $R_{17,18}$ beträgt 340$\Omega$-690$\Omega$. Es wird für diese Schaltung ein 680$\Omega$ Widerstand ausgewählt.
\newpage
\item R13 und R14:\\
Die Spannung die abfallen muss und der Strom der fließt, kann aus dem Datenblatt des Sensors entnommen werden.\\
$U_{F 50mA}=$1,45V \\
$U_{F 30mA}=$1,375V \\
$I_{F}$=30/50mA \\

\begin{center}
$R_{13,14}=\frac{U}{I}=\frac{U_{5V}-U_{F 50mA}}{I_{F}}=\frac{5V-1,45V}{0,05A}=72\Omega$
\end{center}
\begin{center}
$R_{13,14}=\frac{U}{I}=\frac{U_{5V}-U_{F 30mA}}{I_{F}}=\frac{5V-1,375V}{0,03A}=121\Omega$
\end{center}
Da die optische Übertragungsstärke durch den Eingangsstrom bestimmt wird, kann zur vereinfachten Berechnung durch die Kennlinie von Abbildung \ref{Sens_op_if} ein Eingangsstrombereich angenommen werden. Dadurch kann mithilfe der Kennlinie von Abbildung \ref{Sens_vf_if} die abfallende Spannung am Sensor abgelesen werden. Mithilfe dieser Werte kann ein Widerstandsbereich berechnet werden, in dem der Widerstandswert liegen muss, damit die optische Übertragungsstärke  hoch genug ist. Der berechnete Widerstandsbereich von $R_{13,14}$ beträgt 72$\Omega$-121$\Omega$. Es wird für diese Schaltung ein 100$\Omega$ Widerstand ausgewählt.
\item R15 und R16:\\
Diese Widerstandswerte werden mit 22k$\Omega$ angenommen.
\end{itemize}
\subsubsection{Spannungsversorgung}
\begin{figure}[H] 
\begin{center}

\includegraphics[width=15cm]{Bilder/Schaltplan/Spannungsversorgungen}
\caption{Spannungsversorgung Schaltplan}
\label{Spannungsversorgung}

\end{center}
\end{figure}
Die Spannungsversorgung erfolgt mittels eines 12V 400W Schaltnetzteils, welches aus 230V Wechselspannung 12V Gleichspannung erzeugt. Die 5V Versorgung erfolgt mittels eines 5V DC/DC Buck Down Converters.
Die Spannungen werden an die Klemmen X1 und X2 gelegt.

\subsubsection{Raspberry Verkabelung}
\begin{figure}[H] 
\begin{center}

\includegraphics[width=12cm]{Bilder/Schaltplan/Raspberry_Verkabelung}
\caption{Raspberry Schaltplan}
\label{Rasp_Circuit}

\end{center}
\end{figure}
Der Schaltplan zeigt, welche Raspberry-Pins mit welcher Klemme verbunden sind.
Die Pins werden mit den Klemmen X3, X5, X6 verbunden.
Der Raspberry wird über die Klemme X2 versorgt.

\subsubsection{Arduino Verkabelung}
\begin{figure}[H] 
\begin{center}

\includegraphics[width=8cm]{Bilder/Schaltplan/Arduino_Verkabelung}
\caption{Arduino Schaltplan}
\label{Ard_Circuit}

\end{center}
\end{figure}
Der Schaltplan zeigt, welche Arduino Nano-Pins mit welcher Klemme verbunden sind.
Die Pins werden mit der Klemme X4 verbunden.
Der Arduino wird über die Klemme X2 versorgt.

\section{Mechanische Konstruktion}
\subsection{Motorhalterung Futterschüsseldrehplatte}
Der Motor der Drehplatte wird an der Spitze der Welle befestigt. Der Motor wird über die Motoraufnahme mit der Decke der Anlage verschraubt. Damit wird der Motor auf der richtigen Höhe gehalten und ist gegen Drehung gesichert. Die genauen Höhe der Platte kann noch nicht bestimmt werden, da die Gesamthöhe der Anlage und damit die Deckenhöhe nicht festgelegt wurde. Die Position der Befestigung auf der Decke ist noch nicht festgelegt, da die genaue Position der Wände der Anlage noch nicht festgelegt wurden und die Werte von diesen Positionen abhängig sind. Die Motoraufnahme wird mit drei M5 Schrauben an der Decke befestigt. Der Motor wird mit zwei M10 Schrauben mit Muttern an der Motoraufnahme befestigt.
\begin{figure}[H] 
\begin{center}
\includegraphics[width=10cm]{Bilder/Inventor/Motoraufnahme}
\caption{Motoraufnahme}
\label{Motoraufnahme}
\end{center}
\end{figure}
\newpage
\subsection{Motorhalterung Förderband}
Der Motor für das Förderband wird am Ende der Welle befestigt. Der Motor wird auf eine Aufnahme geschraubt und mithilfe eines Winkels an der rechten Stütze des Förderbandes befestigt. Die genaue Position variiert nach Förderbandlänge und muss nach Auslegung dimensioniert werden. Der Winkel wird mit drei M5 Schrauben an die Motoraufnahme geschraubt. Der Motor wird mit zwei M10 Schrauben mit Muttern an der Motoraufnahme befestigt. Der Winkel wird mit vier M6 Schrauben mit Muttern an der Förderbandstütze befestigt.
\begin{figure}[H] 
\begin{center}
\includegraphics[width=9cm]{Bilder/Inventor/Motorhalterung_Foerderband}
\caption{Motorhalterung Förderband}
\label{Motor_mount_foerd}
\end{center}
\end{figure}
\subsection{Sensorhalterung Futterschüsseldrehplatte}

\begin{figure}[H]
\begin{minipage}[t]{6cm}
\vspace{0pt}
\centering
\includegraphics[width=3cm]{Bilder/Inventor/Sensorhalterung_Drehplatte}
\caption{Sensorhalterung Drehplatte}
\label{Sens_Dreh}
\end{minipage}
\hfill
\begin{minipage}[t]{12cm}
\vspace{0pt}
Der Sensor der Drehplatte wird so nah wie möglich auf Höhe der Drehplatte befestigt. Der Sensor wird mit einer Holzleiste direkt auf die Grundplatte geschraubt. Der Sensor wird auf die Holzleiste mit zwei M2,5 Holzschrauben geschraubt und die Leiste wird mit zwei M2,5 Holzschrauben von der Unterseite der Grundplatte auf deren Oberseite verschraubt. Die genaue Position auf der Grundplatte kann noch variieren. Die Höhe der Holzleiste kann sich je nach Höhe der Futterschüsseldrehplatte verändern.
\end{minipage}
\end{figure}
\subsection{Sensorhalterung Förderband}
Der Sensor muss so nah wie möglich an der Klemme auf Höhe der Reflektorfläche befestigt werden. Der Sensor wird mit zwei Holzplatte an der linken Förderbandstütze befestigt. Der Sensor wird auf der ersten Holzleiste mit zwei M2,5 Holzschrauben befestigt. Die erste Holzleiste wird auf der zweiten Holzleiste mit zwei M2,5 Holzschrauben befestigt. Die zweite Holzleiste wird mit zwei M2,5 Schrauben und Muttern an die Förderbandstütze geschraubt. Die genaue Position kann variieren und muss in der Praxis ausgetestet werden. Die genauen Maße der Holzleisten und deren Befestigungen können noch variieren, je nach Größe der Stützen, der Befestigung des Förderbandes und weiteren Faktoren.

\begin{figure}[H]
  \begin{minipage}[hbt]{0.45\textwidth}
    \includegraphics[width=0.9\textwidth]{Bilder/Inventor/Sensorhalterung_Klemme}
 	\caption{Sensorhalterung mit \\Position}
  	\label{Sens_halt_mit_gesamt}
  \end{minipage}
\hspace{.03\linewidth}
  \begin{minipage}[hbt]{0.45\textwidth}
    \includegraphics[width=0.9\textwidth]{Bilder/Inventor/Sensorhalterung_Klemme_2}
  	\caption{Sensorhalterung}
  	\label{Sens_halt_allein}
  \end{minipage}
\end{figure}
\subsection{Halterung Raspberry}
Der Raspberry wird an die Unterseite des 7"  Touchdisplays befestigt. Das Touchdisplay selber wird in eine Vertiefung in der Decke der Anlage geschraubt. 
\subsection{Lüfterhalterung für Raspberry}
Der Lüfter wird am Ende eines Schachts, welcher sämtliche Elektronik, bis auf das 12V Schaltnetzteil, umschließt befestigt. Der Lüfter wird von der Innenseite an ein Loch in der Wand befestigt und von vier M4 Schrauben mit Muttern gehalten. Die genaue befestigung des Kanals und des Lüfters variiert noch je nach verfügbarem Platz. \\
\subsection{Förderband Stützen}

\begin{figure}[H]
\begin{minipage}[t]{6cm}
\vspace{0pt}
\centering
\includegraphics[width=4cm]{Bilder/Inventor/Foerderband_Stuetze}
\caption{Förderband Stütze links}
\label{Foerd_stue}
\end{minipage}
\hfill
\begin{minipage}[t]{11cm}
\vspace{0pt}
Die Halterungen des Förderbandes sind in unserer Konstruktion überdimensioniert gezeichnet, da das Förderband je nach Benutzer eine gewünschte Länge annehmen kann. Die vier Halterungen aus Holz müssen daher standardmäßig eine Förderbandlänge von einem Meter tragen können. Die Halterungen sind mit zwei rechten Winkel an der Grundplatte befestigt. An der Oberseite der vorderen Halterungen ist eine extra Platte angeschraubt, die als Federbefestigung und Halterung der losen Walze dient. Für die Feder wird eine Schraube durch die Platte geschraubt, an der diese eingehängt wird. Für die Halterung der Walze wird ein Bolzen zwischen den beiden gegenüberliegenden Halterungen befestigt, an der zwei kleine Platten die Walzen an einer bestimmten Position halten. Unter der angeschraubten Platte wird ein Lager eingepresst um die Welle zu lagern wie in Abbildung \ref{Stuetz_Platte} zu sehen.  An den beiden hinteren Halterungen wird nur ein Lager für die Welle gepresst wie in Abbildung \ref{Stuetz_ohne_Platte} zu sehen.
\end{minipage}
\end{figure}
\begin{figure}[H]
  \begin{minipage}[hbt]{0.45\textwidth}
    \includegraphics[width=0.9\textwidth]{Bilder/Inventor/Foerderband_mit_Platte}
 	\caption{Stütze mit Platte}
  	\label{Stuetz_Platte}
  \end{minipage}
\hspace{.03\linewidth}
  \begin{minipage}[hbt]{0.45\textwidth}
    \includegraphics[width=0.9\textwidth]{Bilder/Inventor/Foerderband_ohne_Platte}
  	\caption{Stütze ohne Platte}
  	\label{Stuetz_ohne_Platte}
  \end{minipage}
\end{figure}

\newpage
\section{Zusammenfassung und Selbstkritik}
\subsection{Relevante Meilensteine}
01.06.2017 Projektstart \\

13.10.2017 Grundkonzept Freeze\\

08.12.2017 Konstruktion + Elektronik fertig\\

02.03.2018 Dokumentation fertig\\ 
\subsection{Probleme}
\begin{itemize}
\item \textbf{Konstruktion:} Die ersten Konstruktionsvarianten stellten sich als viel zu aufwendig heraus, weshalb die geplante Zeit bis zum Grundkonzept Freeze nicht ausreichte und sich der Grundkonzept Freeze um Monate verschob. Des weiteren haben sich Probleme in der Elektronik ergeben, weshalb sich das Konzept der Elektronik auch nach dem Grundkonzept Freeze verändert hat.
\item \textbf{Elektronik:} Da ich versucht habe, so viel wie möglich von der Ansteuerung selbst zu machen, hat ein Großteil der Schaltung nicht funktioniert und musste verbessert oder ersetzt werden. 
\item \textbf{Dokumentation:} Das Einarbeiten in Latex hat mehr Zeit in Anspruch genommen als erwartet, weshalb sich das Ende der Dokumentation verzögerte.
\end{itemize}
\subsection{Zukünftige Verbesserungen}
\begin{itemize}
\item \textbf{Konstruktion:} Bei erneuter Arbeit an einem Projekt, sollten eher einfache Varianten verwendet werden, die gegebenenfalls noch verbessert und ergänzt werden können, als direkt mit einer schwierigen Variante zu beginnen, welche zu viel Zeit in Anspruch nimmt.
\item \textbf{Elektronik:} Bei der Elektronik muss ich in Zukunft versuchen eher bereits fertige Module für Schaltungen zu verwenden, als diese selbst zu entwerfen. Fertige Module sind meistens günstiger zu erwerben, als die einzelnen Bauteile und verfügen meist über zusätzliche Funktionen, welche die eigene Schaltung nicht bieten könnte. Als Beispiel wird die Motoransteuerung hergenommen. Die fertige Motoransteuerung ist günstiger als die selbst gezeichnete H-Brücke und verfügt zusätzlich über eine Drehzahlregelung mittels \ac{PWM}-Signal, internen Sicherungen gegen Kurzschlüsse und kann die Ströme der Motoren messen.
\item \textbf{Dokumentation:} Ich würde in Zukunft die Dokumentation parallel zur generellen Arbeit im Projekt beginnen, um eine Überforderung zum Schluss des Projektes zu vermeiden.
\end{itemize}
\subsection{Tatsächlicher Zeitpunkt der erreichten Meilensteine}
01.06.2017 Projektstart\\

10.03.2017 Grundkonzept Freeze\\

24.03.2018 Konstruktion + Elektronik fertig\\

27.03.2018 Dokumentation fertig\\
\section{Stückliste}

\begin{table}[htb]
\begin{tiny}
\centering
\begin{tabular}{|p{2cm}|c|c|p{3cm}|p{7cm}|} \hline
Name & ID & Stückzahl & Anmerkung & Link \\ \hline
Optek Optischer Näherungsschalter & OPB732WZ & 2 & - & \url{https://at.rs-online.com/web/p/products/9087113/?gclid=Cj0KCQjw6NjNBRDKARIsAFn3NMq0RvGj3r_5q7xtFIs2eGGmyLAENp-a446vhWDjUIsFB9L3bRjBDXUaAqj1EALw_wcB&cm_mmc=AT-PLA-DS3A-_-google-_-PLA_AT_DE_Automation-_-Sensoren_Und_Messwandler-_-PRODUCT+GROUP&matchtype=&grossPrice=Y&gclsrc=aw.ds} \\ \hline 
Optokoppler & SFH6916 & 2 & 4 Optokoppler pro Bauteil & \url{https://www.neuhold-elektronik.at/catshop/product_info.php?products_id=4181} \\ \hline
Motorsteuerung & VNH2SP30 & 1 & Fertiges Modul aus 2 ICs und zusätzlicher Beschaltung & \url{https://www.ebay.at/itm/30A-VNH2SP30-Dual-Stepper-Motor-Driver-Monster-Moto-Shield-Module-Motortreiber/122543212784?hash=item1c882508f0:g:QGkAAOSwsy9amWPw} \\ \hline
Gleichstrom Motor & FD MY2007 U222 60x & 2 & Schneckengetriebe & \url{https://www.neuhold-elektronik.at/catshop/product_info.php?cPath=96_99&products_id=5522} \\ \hline
DC/DC Wandler 5V & MURATA LSN-5/10-D12-C 5 V-/ 10 A & 1 & 12V zu 5V & \url{https://www.neuhold-elektronik.at/catshop/product_info.php?cPath=222_361&products_id=6778} \\ \hline
Schaltnetzteil 12V & - & 1 & 400W & \url{https://www.amazon.de/dp/B0111MEUWY/ref=twister_B0111MEFW4?_encoding=UTF8&th=1} \\ \hline
Lüfter 12V & EE80251S2-0000-999 & 1 & 80x80x25mm & \url{https://www.conrad.at/de/axialluefter-12-vdc-6286-mh-l-x-b-x-h-80-x-80-x-25-mm-sunon-ee80251s2-0000-999-323905.html} \\ \hline
Widerstand & - & 6 & Kohleschicht 390$\Omega$ & \url{https://www.conrad.at/de/kohleschicht-widerstand-390-axial-bedrahtet-0204-01-w-5-tru-components-1-st-1557169.html} \\ \hline
Widerstand & - & 2 & Kohleschicht 150$\Omega$ & \url{https://www.conrad.at/de/kohleschicht-widerstand-150-axial-bedrahtet-0204-01-w-5-1-st-400157.html} \\ \hline
Widerstand & - & 2 & Kohleschicht 10$\Omega$ & \url{https://www.conrad.at/de/kohleschicht-widerstand-10-axial-bedrahtet-0207-025-w-5-yageo-cfr-25jt-52-10r-1-st-1417641.html} \\ \hline
Widerstand & - & 2 & Kohleschicht 680$\Omega$ & \url{https://www.conrad.at/de/kohleschicht-widerstand-680-axial-bedrahtet-0207-025-w-5-yageo-cfr-25jt-52-680r-1-st-1417677.html} \\ \hline
Widerstand & - & 8 & Kohleschicht 22k$\Omega$ & \url{https://www.conrad.at/de/kohleschicht-widerstand-22-k-axial-bedrahtet-0207-025-w-5-yageo-cfr-25jt-52-22k-1-st-1417666.html} \\ \hline
Arduino Nano & - & 1 & Atmega328P zur PWM generierung & \url{https://www.amazon.de/AptoFun-Org-ATmega328P-FT232RL-Development-kompatibel/dp/B014TE52RS/ref=sr_1_1_sspa?ie=UTF8&qid=1521579402&sr=8-1-spons&keywords=arduino+nano&psc=1} \\ \hline
\end{tabular}
\caption{Stückliste}
\label{Parts list}
\end{tiny}
\end{table}
\newpage
\section{Quellenverzeichnis} 
\begin{table}[htb]
\begin{scriptsize}
\centering
\begin{tabular}{|c|c|p{12cm}|}\hline
Verwendung für Kapitel & Datum & URL \\ \hline
4.3.1.1 & 22.10.2017 & \url{http://tremba.de/orientierung.php}\\ \hline
4.3.1.1 & 22.10.2017 & \url{http://tremba.de/zylindermagnete/zylindermagnete.php}\\ \hline
4.3.1.1 & 22.10.2017 & \url{http://tremba.de/zylindermagnete/db-zylindermagnete-ZMF-2551z.002.pdf}\\ \hline
4.3.1.1 & 22.10.2017 & \url{http://tremba.de/zylindermagnete/db-zylindermagnet-ZMF-1130d.002.pdf} \\ \hline
4.3.1.1 & 22.10.2017 & \url{http://tremba.de/zylindermagnete/db-zylindermagnet-ZMF-1130d.002.pdf} \\ \hline
4.3.1.1 & 22.10.2017 & \url{http://tremba.de/kurzhubmagnete/db-hubmagnete-KHM-1113.001.pdf} \\ \hline
4.3.1.1 & 22.10.2017 & \url{http://tremba.de/hubmagnete/db-hubmagnete-HMF-3830d-15.002.pdf} \\ \hline
4.3.2, 4.4.2, 4.4.3.7 & 25.10.2017 & \url{http://at.rs-online.com/web/p/products/9087113/?grossPrice=Y&cm_mmc=AT-PLA-DS3A-_-google-_-PLA_AT_DE_Automation-_-Sensoren_Und_Messwandler-_-PRODUCT+GROUP&matchtype=&gclid=Cj0KCQjw6NjNBRDKARIsAFn3NMq0RvGj3r_5q7xtFIs2eGGmyLAENp-a446vhWDjUIsFB9L3bRjBDXUaAqj1EALw_wcB&gclsrc=aw.ds} \\  \hline
4.3.2, 4.4.2, 4.4.3.7 & 25.10.2017 & \url{http://docs-europe.electrocomponents.com/webdocs/14a6/0900766b814a6875.pdf} \\ \hline
4.3.2, 4.4.2, 4.4.3.7 & 25.10.2017 & \url{http://www.mouser.com/ds/2/414/OP265-266-45966.pdf} \\ \hline
4.4.3.1 & 01.12.2017 & \url{http://forum.arduino.cc/index.php?topic=430436.0} \\ \hline
4.4.3.1 & 01.12.2017 & \url{http://www.bristolwatch.com/ele/h_bridge.htm} \\ \hline
4.4.3.1 & 01.12.2017 & \url{https://www.conrad.at/de/mosfet-vishay-irf9630pbf-1-p-kanal-74-w-to-220-162541.html?insert=U3&gclid=Cj0KCQiAmITRBRCSARIsAEOZmr5pGS-Z97ve6qOgyJbyrJljLlzOkaKVUCa1omiA_ZBUXX-yv1mTz2gaAhY5EALw_wcB} \\ \hline
4.4.3.1 & 01.12.2017 & \url{https://www.conrad.at/de/mosfet-infineon-technologies-irf630n-1-n-kanal-82-w-to-263-3-162421.html} \\ \hline
4.4.3.1 & 14.02.2018 & \url{http://www.produktinfo.conrad.com/datenblaetter/150000-174999/162421-da-01-en-IRF_630_N.pdf} \\ \hline
4.4.3.1 & 14.02.2018 & \url{http://www.produktinfo.conrad.com/datenblaetter/150000-174999/162541-da-01-en-TRANSISTOR_HEXFET_IRF9630PBF_TO_220__VIS.pdf} \\ \hline
4.4.3.2, 4.4.3.6 & 02.03.2018 & \url{https://www.ebay.at/itm/30A-VNH2SP30-Dual-Stepper-Motor-Driver-Monster-Moto-Shield-Module-Motortreiber/122543212784?hash=item1c882508f0:g:QGkAAOSwsy9amWPw} \\ \hline
4.4.3.2, 4.4.3.6 & 02.03.2018 & \url{http://www.st.com/content/ccc/resource/technical/document/datasheet/group2/66/b8/f5/2c/9a/66/41/c7/CD00043711/files/CD00043711.pdf/jcr:content/translations/en.CD00043711.pdf} \\ \hline
4.4.3.2, 4.4.3.6 & 02.03.2018 & \url{http://www.instructables.com/id/Monster-Motor-Shield-VNH2SP30/} \\ \hline
4.4.3.3, 4.4.3.10 & 04.03.2018 & \url{http://ww1.microchip.com/downloads/en/DeviceDoc/Atmel-42735-8-bit-AVR-Microcontroller-ATmega328-328P_Datasheet.pdf} \\ \hline
4.4.3.4, 4.4.3.6, 4.4.3.7 & 11.12.2017 & \url{https://www.neuhold-elektronik.at/datenblatt/N6465.pdf} \\ \hline
4.4.3.8 & 12.03.2018 & \url{https://www.neuhold-elektronik.at/catshop/product_info.php?cPath=222_361&products_id=67786} \\ \hline
\end{tabular}
\caption{Quellen}
\label{Quellen}
\end{scriptsize}
\end{table}