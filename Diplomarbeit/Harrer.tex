\chapter{Java-Programm}
\label{sec:java-programm}

\section{Anforderungen}
\subsection{Programm}
Die Hauptaufgabe des Programms ist es dem Benutzer eine Möglichkeit zum Steuern der Katzenfütterungsanlage zu Verfügung zu stellen. Weiters soll das Programm die Motoren steuern und die Sensoren in der Anlage auswerten können. Für diese Aufgabe sollen die IO-Pins am Raspberry verwendet werden.
\subsection{Design - Benutzerinterface}
Das Design der GUI-Fenster soll einfach und übersichtlich gestaltet werden. Auf der Hauptseite, also der Seite die immer zu sehen ist, sollen Informationen dargestellt werden, die dem Benutzer einen schnellen Überblick über den Zustand der Anlage geben. Alle anderen nicht direkt ersichtlichen Funktionen sollen über sinnvoll benamte Menüpunkte erreichbar sein.
Der Benutzer soll mit Hilfe eines kleinen Touchdisplays die Möglichkeit haben die Anlage zu steuern. Deswegen muss darauf geachtet werden, dass alle GUI-Fenster sinnvoll per Touch-Gesten verwenbar sind.
\subsection{Externe Steuerung}
Da die Katzenfütterungsanlage die Katze füttern soll, wenn die Familie der Katze auf Urlaub ist, sollte die Anlage auch über das Internet erreichbar sein. Dafür gibt es die Möglichkeit einen Benutzer auf der Anlage anzulegen mit welchem man anschließend über eine Webseite auf die Anlage zugreigen kann. 

\newpage

\section{Voruntersuchung}
\subsection{Wieso Java und nicht C?}
Zu Beginn musste entschieden werden mit welcher Programmiersprache gearbeitet werden soll. Zur Auswahl standen Java und C.
Vorteile von C:
\begin{itemize}
\item[1] Echtzeitfähige Steuerung der Motoren und Senosren
\item[2] Hardwarenahe Programmierung für die Pins
\end{itemize}
Vorteile von Java:
\begin{itemize}
\item[1] Erstellen einer GUI ist einfacher
\item[2] Implementieren eines Servers ist einfacher
\end{itemize}

Nach dem Gegenüberstellen der Vorteile wurde Java als Programmiersprache gewählt.

\subsection{Wieso das Raspberry Pi 3 Model B?}
Schon zu Beginn der Diplomarbeit war klar, dass mit deinem Raspberry gearbeitet werden soll. Nun musste entschieden werden welches Model verwendet werden soll. Wir haben das Raspberry Pi 3 Model B aufgrund folgender technischer Daten gewählt:
\begin{itemize}
\item[1] Rechenleistung
\item[2] Anzahl der GPIO-Pins
\item[3] WLAN-Fähigkeit
\end{itemize}

\subsection{Auswahl eines Touchdisplays}
Das Display muss folgendet Anforderungen erfüllen:
\begin{itemize}
\item[1] Es muss ein Touchscreen-Display sein
\item[2] Es muss einfach an das Raspberry anschließbar sein
\item[3] Es sollte nicht zu teuer sein
\end{itemize}

Aufgrund dieser Anforderungen wurde das Touchdisplay von Raspberry gewählt.

\subsection{Wieso pi4j?}
Da Java als Programmiersprache gewählt wurde, musste eine Möglichkeit die GPIO-Pins anzusteuern gefunden werden.
Da bei der Recherche außer pi4j Java kaum etwas gefunden wurde, wurde pi4j gewählt. Weiters vorteilhaft ist, dass das Ansteuern der Pins via Code nicht sehr kopliziert ist. 

\subsection{Wieso Mongodb?}
Eine Datenbank wurde gewählt, weil es gegebüber des Speichers der Daten in eine Datei mehrere Vorteile aufweist. 
\subsubsection{Vorteil gegenüber Daten in Datei speichern}
Vorteile einer Datenbank:
\begin{itemize}
\item[1] Keine Probleme mit Pfaden
\item[2] Daten sind alle in einem Punkt gespeichert und nicht im System verteilt
\item[3] Der benötigte Code für die Datenbank macht das Programm übersichtlicher
\end{itemize}
\subsubsection{Vergleich mit anderen Datenbanken}
Vorteile von Mongodb gegenüber anderen Datenbanken (zB mySQL):
\begin{itemize}
\item[1] Mongodb ist schemenlos (Daten benötigen keine bestimmte Struktur)
\item[2] Mongodb ist kostenfrei
\end{itemize}

\subsection{Kommunikation mit der Web-Applikation}
Der Server mit dem die Web-Applikation kommunizieren kann, wird aufgrund der gewählten Programmiersprache, in Java geschrieben. Der Server wird im Hintergrund aktiv sein und auf Anfragen der Web-Applikation warten. Je nach Anfrage wird der Server Daten zurück senden oder Methoden im Programm aufrufen. Die Daten, die bei der Kommunikation ausgetauscht werden, haben den Datentyp JSON.

\newpage

\section{Umsetzung}
\subsection{Mongodb}
\subsubsection{Allgemeines}
\subsubsection{Datenbankmanagementsystem DBS}
\subsubsection{Code Beispiele}
\subsection{pi4j}
\subsubsection{Allgemeines}
\subsubsection{Pin Numbering Sheme}
\subsubsection{Gewählte Pin Belegung}
\subsubsection{Wieso Sigleton?}
\subsubsection{Code Beispiele}
\subsection{Server-Client}
\subsubsection{Übertragungsprotokoll}
\subsubsection{Wie wird geantwortet?}
\subsubsection{Was wird am Raspberry vom Java-Programm gemacht?}
\subsection{GUI-Fenster}
\subsubsection{MainWindow}
\subsubsection{TimeManagement}
\subsubsection{CreateUser}
\subsubsection{ManualControl}
\subsubsection{Positionsinformation}
\subsubsection{SystemInfo}
\subsubsection{Update}

\section{Zusammenfassung/Verbesserungsmöglichkeiten}
\subsubsection{Probleme mit Mongodb am Raspberry}
\subsubsection{GUI auf "Touchscreen-Design" abändern}
\subsubsection{Besser Benutzerverwaltung - Mehrere Benutzer anlegen}
\subsubsection{Selbst erstellbare Vorlagen in denen Zeiten gespeichert werden}
