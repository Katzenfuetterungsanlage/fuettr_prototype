\lohead{Florian Greistorfer}
\chapter{Webserver und Client}

\section{Begriffserklärungen}

\subsection{Server}
Ein Computer oder Programm, der oder das Zugriff auf eine Resource oder einen Dienst in einem Netzwerk ermöglicht

\subsection{Client}
Ein Computer oder Programm, der oder das auf einen Server Zugreift

\section{Anforderungen}

\subsection{Webserver}
Auf der Katzenfütterungsanlage läuft ein Webserver, der es ermöglicht, dass der Benutzer das Gerät über das Internet erreichen kann. Hauptaufgaben des Servers sind dabei, Daten bereitzustellen, zu verabeiten und zu speichern und den Webclient zur Verfügung zu stellen.

\subsection{Client}
Der Client soll dem Benutzer ermöglichen, die Katzenfütterungsanlage über einen Webbrowser zu steuern. Ein Benutzername und ein Passwort sind erforderlich, damit man das Gerät bedienen kann. Das Design soll eindeutig und übersichtlich gehalten sein. Auf der Startseite sollen die eingestellten Fütterungszeiten zu sehen sein und eine allgemeine Übersicht. Über eine Navigationsleiste sollen die weiteren Funktionen erreichbar sein. Auf einer Seite sollen die Fütterungszeiten einstellbar sein und der Fütterungszyklus ein und ausschaltbar sein. Auf einer anderen Seite, sollen die aktuellen Zustände der Motoren und Sensoren angezeigt werden. Zusätzlich soll es noch eine Updatefunktion geben. Auf der letzten Seite sollen Geräteinformationen wie die aktuelle Softwareversion angezeigt werden.

\section{Voruntersuchung}

\subsection{Typescript}
Typescript ist eine Weiterentwicklung der Sprache Javascript, die strenge Datentypen hat. Typescript muss von einem Transpiler (=Übersetzer) in Javascript übersetzt werden. Javascript kann direkt von jedem herkömmlichen Browser ausgeführt werden.

\subsection{Node.js}
Node.js ist eine Laufzeitumgebung, die es ermöglicht, dass Javascript direkt auf einem Rechner ausgeführt werden kann. 

\subsection{Angular 2/4}
Angular  

\subsection{Bootstrap}
Bootstrap

\section{Umsetzung}

\subsection{Client}

\subsubsection{Design}

\subsubsection{Funktion}

\subsection{Server}

\subsubsection{Funktion}

\subsubsection{Mongodb}

\subsubsection{Kommunikation mit dem Java Programm}

\section{Zusammenfassung und Verbesserungsmöglichkeiten}